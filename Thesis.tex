\documentclass[10pt,a4paper]{article}
\usepackage[utf8]{inputenc}
\usepackage{amsmath}
\usepackage{amsfonts}
\usepackage{amssymb}
\usepackage{amsthm}
\usepackage{natbib}
\usepackage{algorithm}
\usepackage{algpseudocode}
\usepackage{titlesec}

\setcounter{secnumdepth}{4}

\usepackage[mathscr]{euscript}
\usepackage{tikz}

\newcommand{\Cross}{\mathbin{\tikz [x=1.8ex,y=1.8ex,line width=.15ex] \draw (0,0) -- (1,1) (0,1) -- (1,0);}}

\newcommand*{\N}{\mathbb{N}}
\DeclareMathOperator*{\argminA}{arg\,min}

\newcommand{\mathDef}{\overset{\textit{def}}{=}}

\usepackage{hyperref}
\hypersetup{
    colorlinks=true, %set true if you want colored links
    linktoc=all,     %set to all if you want both sections and subsections linked
    linkcolor=black,  %choose some color if you want links to stand out
}

\theoremstyle{definition}

\newtheorem{definition}{Definition}[section]
\newtheorem*{remark}{Remark}

\begin{document}
	
\bibliographystyle{plain}
\bibliography{Bibliography.bib}


\tableofcontents
\newpage

fdsfsfsdfsdfsd
f
sd
fsd
f
sdfsd
fsdfsdtertertertereterterterterterterterter

\begin{equation}
\begin{array} {lllrr} 
\text{min} & \displaystyle\sum_{n = 1}^N \sum_{k = 1}^K \sum_{x \in C} g(x)^{f_k^i} Y_x^{f_k^i} \\\\
& \displaystyle\sum_{n = 1}^N \sum_{k = 1}^K \sum_{x \in C} T(x)^{f_k^i} Y_x^{f_k^i} & \leq &  D_{\textbf{F}} \\
& \displaystyle\sum_{n = 1}^N \sum_{k = 1}^K \sum_{x \in C} Y_x^{f_k^i} & = & 1 \\\\
& Y_x^{f_k^i} \in \lbrace 0, 1 \rbrace &&
\end{array}
\end{equation}


\begin{equation}
\begin{array} {lllrr} 
\text{min} & \displaystyle\sum_{\pi \in \Pi} \sum_{n = 1}^N \sum_{k = 1}^K \sum_{x \in C} g(x)^{f_k^i} Y_{(\pi, x)}^{f_k^i} \\\\
& \displaystyle\sum_{n = 1}^N \sum_{k = 1}^K \sum_{x \in C} T(x)^{f_k^i} Y_x^{f_k^i} & \leq &  D_{\textbf{F}} \\
& \displaystyle\sum_{\pi \in \Pi} \sum_{n = 1}^N \sum_{k = 1}^K \sum_{x \in C} Y_{(\pi, x)}^{f_k^i} & = & 1 \\\\
& Y_{(\pi, x)}^{f_k^i} \in \lbrace 0, 1 \rbrace &&
\end{array}
\end{equation}


\newpage

\section{Serverless Computing Paradigm}

In serverless computing platforms, computation is done by so-called \textit{function instances} which are completely managed by the serverless computing platform provider and act as tiny servers that can be invoked based on events forwarded by end-users \cite{PMSCP}.

Serverless computing platforms handle almost every aspect of the system administration tasks needed to deploy a workload on the cloud, providing a new simplified programming model according to which developers can focus on the business aspects of their applications only \cite{COSE}.

Moreover, the paradigm lowers the cost of deploying applications too, adopting a so-called “\textit{pay as you go}” pricing model, by charging for execution time rather than for allocated resources \cite{COSE}.


\subsection{Serverless Function}

A \textit{serverless function} represents a stateless, event-driven, self-contained unit of computation implementing a business functionality.

Although a serverless function generally represents a unit of executable code, submitted to FaaS platforms by developers using one or a combination of the programming languages supported by FaaS providers, a serverless function can be any cloud services eventually necessary to business logic, like cloud storage, message queue services, pub/sub messaging service etc.

When it represents executable code, developers can specify several configuration parameters, like timeout, memory size or CPU power \cite{COSE}.

A serverless function can be triggered through events or HTTP requests following which the FaaS provider executes it in a containerized environment, like container, virtual machine or even processes, using the specified configuration.

\subsection{Serverless Application}

A \textit{serverless application} represents a stateless and event-driven software system made up of a set of serverless functions hosted on one or more FaaS platforms and combined together by an \textit{orchestrator}, which handles events among functions and triggers them in the correct order according to the business logic defined by developers. 

\section{Orchestrator Model}

\subsection{Resource owner}

A \textit{resource owner}, henceforward denoted with $R$, represents an entity capable of \textit{creating}, \textit{modifying} and \textit{authorizing} access to several resources of our system.

\section{Serverless Choreography}

According to our framework, a \textit{serverless choreography} is a \textit{resource} used to model and implement both serverless functions and serverless applications.

Informally, a serverless choreography is derived by the control-flow graph which describes, using graph notation, all paths that might be traversed through a serverless application during its execution. Similarly, a serverless choreography describes calling relationships between serverless functions which can be combined using several types of control-flow connectors.

\subsection{Definition}

Formally, let $n \in \N \setminus \left\lbrace 0 \right\rbrace $, being $R$ the resource owner, a \textit{serverless choreography}, or simply \textit{choreography}, $\mathcal{S}_R$ owned by $R$ is a weighted directed graph defined as follows:

\begin{equation}
	\mathcal{S}_R \mathDef (\Phi,E)
\end{equation}

where:

\begin{itemize}
	\item $\Phi$ represents a finite set of vertices such that $|\Phi|=n$. 
	
	Each vertex $\phi \in \Phi$ is called \textit{abstract serverless function} which represents, generally, a computational unit; we will describe more in detail what we mean by abstract serverless function shortly;
	
	\item $E \subseteq \Phi \times \Phi $ represents a finite set of directed edges.
	
	For all $i,j \in \left[ 1, n \right]$, each directed edge from $\phi_i \in \Phi$ to $\phi_j \in \Phi$, denoted as $\left( \phi_i, \phi_j \right)$, indicates that the abstract serverless function $\phi_j$ \textit{can} be called by $\phi_i$, representing the interaction between $\phi_i$ and $\phi_j$ defined by the business logic;
	
	\item For all $i,j \in \left[ 1, n \right]$, the number $p_{ij} \in \left[ 0, 1 \right]$ is the weight assigned to the edge $\left(\phi_i, \phi_j \right)$ and represents the \textit{transition probability} from $\phi_i$ to $\phi_j$, that is the probability of invoking $\phi_j$ after finishing the execution of $\phi_i$.
	
	$P : \Phi \times \Phi \to \left[ 0, 1 \right]$ is defined as the \textit{transition probability function} according to which $P\left(\phi_i, \phi_j \right) = p_{ij}$. Clearly, when $P\left(\phi_i, \phi_j \right) = 0$, the directed edge $\left( \phi_i, \phi_j \right)$ does not exist according to business logic, therefore $\phi_i$ cannot invoke $\phi_j$;
	
\end{itemize} 

A choreography $\mathcal{S}_R$ can be uniquely identified by an ordered pair $(a, b)$, where $a$ is the name of the resource owner $R$, while $b$ is the function choreography name.

Clearly, we say that choreography models a serverless function when $|\Phi| = 1$ and $|E| = 0$; conversely, it models a serverless function application when $|\Phi| > 1$ and $|E| > 0$.

From now, a choreography $\mathcal{S}_R$ will be briefly denoted by $\mathcal{S}$ when no confusion can arise about the resource owner $R$.

\subsubsection{Notations}

Let $\mathcal{S} = (\Phi,E)$ a choreography. Let's now introduce following notations and functions:

\begin{itemize}
	\item A \textit{path} $\pi$ is defined as a finite sequence of distinct vertices and edges as follows:
	
	\begin{equation}
	\pi = \phi_1 e_1 \phi_2 e_2 \ldots e_{n-2}\phi_{n-1} e_{n-1} \phi_n
	\end{equation}
	 
	where:
	
	\begin{itemize}
		\item $\phi_i \in \Phi$, for all $i \in \left[ 1, n \right]$
		\item $e_i = \left( \phi_i, \phi_{i+1} \right) \in E$, for all $i \in \left[ 1, n-1 \right]$
	\end{itemize}
	
	\item Given a path $\pi$, we define the \textit{transition probability} of the path $\pi$ the following quantity:
	
	\begin{equation}
		TPP(\pi) = \prod_{i = 1}^{n-1} P\left(\phi_i, \phi_{i+1} \right)
	\end{equation}

	\item Let $\phi_i \in \Phi$ to $\phi_j \in \Phi$ for all $i,j \in \left[ 1, n \right]$, the set $\Pi(\phi_i, \phi_j)$ identifies all possible paths starting from vertex $\phi_i$ and ending at vertex $\phi_j$.
	
	\item The set $out(\phi_u)$ ($in(\phi_u)$) denotes all edges starting (ending) from (to) vertex $\phi_u$, while the set $succ(\phi_u)$ ($pred(\phi_u)$) includes all direct successor (predecessors) vertices of $\phi_u$. Formally:
	
	\begin{eqnarray}\label{outDef}
		out(\phi_u) & \mathDef & \left\lbrace (\phi_u, \phi) \in E, \quad \forall \phi \in \Phi  \right\rbrace \\
		in(\phi_u) & \mathDef & \left\lbrace (\phi, \phi_u) \in E, \quad \forall \phi \in \Phi  \right\rbrace \\
		succ(\phi_u) & \mathDef & \left\lbrace \phi \in \Phi \mid (\phi_u, \phi) \in out(\phi_u)  \right\rbrace \\
		pred(\phi_u) & \mathDef & \left\lbrace \phi \in \Phi \mid (\phi, \phi_u) \in in(\phi_u)  \right\rbrace 
	\end{eqnarray}
	
	\item $NI: \Phi \to \left[ 0, \infty \right)$ is a function representing the average number of invocations of an abstract serverless function $\phi \in \Phi$ during the execution of a serverless choreography $\mathcal{S}$.
	
	\item $D : \Phi \times \Phi \to \left[ 0, \infty \right)$ represents a delay function according to which $D\left(\phi_i, \phi_j \right)$ identifies the delay from $\phi_i$ to $\phi_j$ due to network delay or orchestration task;  
	
\end{itemize}

\subsubsection{Abstract Serverless Function}

Supposing that a choreography $\mathcal{S} = (\Phi,E)$ is given, an \textit{abstract serverless function} $\phi \in \Phi$, or simply \textit{abstract function}, is a \textit{resource} that represents a computational unit needed by business logic. 

According to our model, there are two types of abstract functions implementations:

\begin{itemize}
	
	\item $\phi$ is defined as \textit{serverless executable functions}, or simply \textit{executable function}, when $\phi$ contains executable code which implementing business logic.
	
	$\mathscr{F_E}$ is defined as the set containing all executable function of $\mathcal{S}$ and it is formally defined as follows:
	
	\begin{equation}
		\mathscr{F_E} \mathDef \left\lbrace \phi \in \Phi \mid \phi \text{ is a serverless executable function }\right\rbrace 
	\end{equation}
	
	However, multiple different implementations of a given executable function can be provided by developers which, although they must be semantically and logically equivalent, may eventually expose different performance or cost behavior. 
	
	Therefore, for each $\phi \in \mathscr{F_E}$, the set $\textbf{F}_{\phi}$, containing all implementations of $\phi$, generally denoted by $f_{\phi}$, such that $|\textbf{F}| \geq 1 $, must exist.

	We will explain later that the aim of our framework is to pick exactly one $f_{\phi} \in \textbf{F}_{\phi}$ whose properties allow us to meet user-specified QoS objective.
	
	We assume that every $f_{\phi} \in \textbf{F}_{\phi}$ for all $\phi \in \mathscr{F_E}$ is already deployed on one or more FaaS platform by developers.
		
	\item $\phi$ is called \textit{serverless orchestration functions}, or \textit{orchestration functions}, when $\phi$ contain orchestration code needed to decide which and how the next abstract functions must be called; in other words, orchestration code is control-flow code.
	
	$\mathscr{F_O}$ is defined as the set containing all executable function of $\mathcal{S}$ and it is formally defined as follows:
	
	\begin{equation}
		\mathscr{F_O} \mathDef \left\lbrace \phi \in \Phi \mid \phi \text{ is a serverless orchestration function }\right\rbrace 
	\end{equation}

	
\end{itemize}

Clearly, based on above definition, we can say: 

\begin{eqnarray}
\mathscr{F_E} \cap \mathscr{F_O} & = & \emptyset \\
\mathscr{F_E} \cup \mathscr{F_O} & = & \Phi \\
|\mathscr{F_E}| + |\mathscr{F_O}| &=& |\Phi| 
\end{eqnarray}

% Generally it contains the description of what the corresponding implementation does including several proprieties, like input or output data types.

According to the serverless paradigm, the execution of a serverless application starts with a particular function, which we will denote as the \textit{entry point}, while any other serverless functions, belonging to the application, will be invoked subsequently according to specified business logic. 

Clearly, the execution of a serverless application ends when ends the execution of the last function according to the business logic. That ending function will be denoted as \textit{end point}. According to our point of view, the end point is not unique.

Obliviously, in order to appropriately model a serverless application, any choreography $\mathcal{S}$ must have exactly one entry point and at least one end point. 

Given $\mathcal{S}$ a choreography, the entry point of $\mathcal{S}$ is denoted by $\phi_{entry}$ while the end point by $\phi_{end}$. Formally:

\begin{eqnarray}
	\phi_{entry} \in \Phi \text{ is the entry point of } \mathcal{S} & \Leftrightarrow & in(\phi_{entry}) = \emptyset \\
	\phi_{end} \in \Phi \text{ is the end point of } \mathcal{S} & \Leftrightarrow & out(\phi_{end}) = \emptyset
\end{eqnarray}

\subsection{Configuration}

Given a choreography $SC_R = (\Phi,E)$, the basic goal of our framework is to determine the so-called \textit{serverless choreography configuration}, or simply \textit{configuration}, of $SC_R$ that allows us to meet user-specified QoS objectives.

\subsubsection{Serverless Functions Configuration}

Let an abstract executable serverless function $\mathscr{F_E}$ and \textbf{F} the corresponding set of serverless function which implement it, we define the following vector:

\begin{equation}
	x \in \textbf{F} \times \N
\end{equation}

as the \textit{serverless functions configuration}, or simply \textit{configuration}, of $\mathscr{F_E}$. In this context, for any $f \in \textbf{F}$ and $m \in \N$, given $x =(f,m)$, $f$ represents a particular serverless function while $m$ represents the allocated memory size.

At this point, we can define some useful functions:

\begin{itemize}	
	\item $C_{\textbf{F}} : \textbf{F} \times \N \to \left[ 0, \infty \right)$ which represents a cost function for serverless functions. For any $f \in \textbf{F}$ and $n \in \N$, $C_{\textbf{F}}(f,n)$ returns the average cost paid by developers to execute $f$ using an allocated memory size equal to $n$.
	
	\item $RT_{\textbf{F}} : \textbf{F} \times \N \to \left[ 0, \infty \right)$ which represents a delay function for serverless functions. For any $f \in \textbf{F}$ and $n \in \N$, $RT_{\textbf{F}}(f,n)$ returns the average response time of function $f$ using $n$ as allocated memory size.
\end{itemize}

Any abstract serverless function is uniquely identified by an ordered string tuple $(a, b, c)$, where:
\begin{itemize}
	\item $a$ represents the name of the resource owner $R$;
	\item $b$ represents the name of the serverless choreography $SC_R$;
	\item $c$ is the name of the abstract serverless function $\phi$
\end{itemize}

\subsubsection{Serverless Choreography Configuration}

Let $n,k \in \N$, such that $k \leq n$, and $SC_R$ a serverless choreography where $|\Phi| = n$ and $|\mathscr{F_E}| = k$. 

The basic goal of our framework is to determine a so-called \textit{serverless choreography configuration} that allows us to meet user-specified QoS objectives. Formally, a serverless choreography configuration $\textbf{x}$ is a $k\times2$ matrix such that:

\begin{equation}
\textbf{X} \in \lbrace \textbf{F}_1 \times \mathbb{N} \rbrace \times \ldots \times \lbrace \textbf{F}_k \times \mathbb{N} \rbrace = \Cross_{i = 1}^k \lbrace \textbf{F}_i \times \mathbb{N} \rbrace
\end{equation}

In other words, $\textbf{x}$ is a matrix where, for $i \in \left[ 1, k \right]$, the $i$-th row contains a serverless functions configuration for the abstract executable serverless functions $F_i \in \mathscr{F_E}$. 

More precisely, for $v \in \left[ 1, k \right]$, the matrix $\textbf{X}$ is such that:

\begin{itemize}
	\item the element $(v,1)$ identifies the serverless function $f \in \textbf{F}_v$ which implements the abstract executable serverless function $F_v \in \mathscr{F_E}$;
	
	\item the element $(v,2)$ identifies the allocated memory size $m \in N$ of the serverless function identified by the element $(v,1)$.
\end{itemize}

\subsection{Metrics}

Let $\mathcal{C} = (\Phi,E)$ a choreography such that:
\begin{itemize}
	\item $\mathcal{C}$ is a DAG without any cycles and loops;
	\item Is true that the transition probabilities of all paths between the start point and
	the end point of $\mathcal{C}$ can sum up to $1$:
	
	\begin{equation}
		\sum_{\pi \in \Pi(\phi_{start}, \phi_{end})} TPP(\pi) = 1
	\end{equation}

\end{itemize}

At this point, let $\pi = \phi_1 e_1 \phi_2 e_2 \ldots e_{n-2}\phi_{n-1} e_{n-1} \phi_n$ a path. We define the response time of $\pi$ as follows:

\begin{equation}
	RT_P(\pi, \textbf{X}) = \sum_{\substack{1\le i\le n\\ \phi_i \in \mathscr{F_E}}} N(\phi_i) \cdot RT_{\textbf{F}} (X_{\phi_{i}1},X_{\phi_{i}2}) + \sum_{i = 1}^{n-1} N(\phi_i) \cdot D(\phi_i,\phi_{i+1})
\end{equation}

Let $\textbf{X}$ a configuration of the choreography $\mathcal{C}$, the response time of $\mathcal{C}$ is defined as follows:

\begin{equation}
	RT_C(\mathcal{C}, \textbf{X}) = \sum_{\pi \in \Pi(\phi_{start}, \phi_{end})} TPP(\pi) \cdot RT_{P}(\pi, \textbf{X})
\end{equation}

\subsection{Structures}

In this section, we will define several types of structures belonging to a given serverless choreography. 

Let $SC_R = (\Phi,E)$ a serverless choreography. We say that a \textit{serverless sub-choreography} $SC_R'$, or simply \textit{sub-choreography}, is a weighted directed sub-graph of $SC_R$. Formally:

\begin{equation}
	SC_R' \mathDef (\Phi',E') \qquad \text{ where } \Phi' \subseteq \Phi \wedge E' \subseteq E
\end{equation}

We denote $\phi_{start}' \in \Phi'$ and $\phi_{end}' \in \Phi'$, respectively, as the \textit{entry point} and the \textit{end point} of the sub-choreography $SC_R'$

\subsubsection{Parallel}

Supposing that a choreography $SC_R = (\Phi,E)$ is given, let us consider two vertices $\phi_i \in \Phi$ and $\phi_j \in \Phi$, for any $i,j \in \left[ 1, n \right]$ such that $i \neq j$.

We say that the sub-choreography $\mathcal{P}$, such that $\phi_{start}' = \phi_i$ and $\phi_{end}' = \phi_j$, is called \textit{parallel structure} when:

\begin{equation}
	TPP(\pi) = 1 \qquad \forall \pi \in \Pi(\phi_{start}^*, \phi_{end}^*)
\end{equation}

Let $out(\phi_{start}^*)$ the set of all edges starting from vertex as defined in equation \ref{outDef} and suppose that $|out(\phi_{start}^*)| = s \in \N \setminus \left\lbrace 0 \right\rbrace $.

Then $\Theta$ is defined as the set, such that $|\Theta| = s$, containing all the sub-choreographies $(\Phi^{**}, E^{**})$ which will be executed at the same time during the processing of the parallel structure. More precisely:

\begin{equation}
	\begin{array}{c}
	\Theta \mathDef \left\lbrace (\Phi^{**}_1, E^{**}_1), \ldots ,(\Phi^{**}_s, E^{**}_s) \right\rbrace \\\\
	\text{ where } \\\\
	\phi_{start_{i}}^{**}, \phi_{end_{i}}^{**} \in \Phi^{**}_i \\\\
	
	\phi_{start_{i}}^{**} \in succ(\phi_{start}^*) \wedge \phi_{end_{i}}^{**} \in pred (\phi_{end}^*) \quad \forall i \in \left[ 1, s \right]
	\end{array}
\end{equation} 

Simply the response time of a parallel structure is the longest response time of all
sub-choreographies $(\Phi^{**}, E^{**})$ in it, while its cost is equal to the sum of costs of execution of all sub-choreographies. Formally, being $\textbf{X}$ a configuration of the choreography $SC_R$:

\begin{equation}
	RT_{parallel}(\mathcal{P}, \textbf{X}) = max \left\lbrace RT_C(\mathcal{\theta}, \textbf{X}) \mid \theta \in \Theta \right\rbrace 
\end{equation}

\begin{equation}
	C_{parallel}(\mathcal{P}, \textbf{X}) = \sum_{\theta \in \Theta} C_C(\theta, \textbf{X})
\end{equation}

\subsubsection{Branch}

We say that A sub-choreography $\mathcal{B}$, such that $\phi_{start}' = \phi_i$ and $\phi_{end}' = \phi_j$, is called \textit{branch structure} when:

\begin{equation}
	\begin{array}{c}
	TPP(\pi) \neq 1 \qquad \forall \pi \in \Pi(\phi_{start}^*, \phi_{end}^*) \\
	\wedge \\
	\displaystyle\sum_{\pi \in \Pi(\phi_{start}^*, \phi_{end}^*)} TPP(\pi) = 1
	\end{array}
\end{equation}


\section{Optimization}


\subsection{Performance Modeling}






We define the utility function as follows:

\begin{equation}
	F(\textbf{x}) \mathDef w_{RT} \cdot \dfrac{RT_{max} - RT(\textbf{x})}{RT_{max} - RT_{min}} + w_{C} \cdot \dfrac{C_{max} - C(\textbf{x})}{C_{max} - C_{min}}
\end{equation}

where $w_{RT}, w_{C} \geq 0$, $w_{RT} + w_{C} = 1$, are weights for the different QoS attributes, while $RT_{max}$ ($RT_{min}$) and $C_{max}$ ($C_{min}$) denote, respectively, the maximum (minimum) value for the overall expected response time and cost.






Before to proceed, when the serverless workflow configuration $x$ is used,




\begin{itemize}
	\item $D(\textbf{x})$ denotes the end-to-end response time of current serverless workflow $G_s$ when the serverless workflow configuration $x$ is used. At the same time, $C(\textbf{x})$ represents the  
	\item $C(\textbf{x})$ re
\end{itemize}

gdfgdfgdfgd

\begin{itemize}
	\item $\textbf{x}(\mathscr{F})$ is used to denote the configuration $\left( f, m \right) $ of the serverless abstract function $\mathscr{F}$.
	\item $C_{func}(\textbf{x}(\mathscr{F}))$ represents the cost to pay to execute the concrete serverless function $f$ with memory size $m$.
\end{itemize}



\begin{equation}
	C_{walk}(w, \textbf{x}) = \sum_{\mathscr{F} \in w} C_{func}(\textbf{x}(\mathscr{F}))
\end{equation}


\begin{equation}
	C_{conf}(\textbf{x}) = \sum_{w \in SP_{G_s}(\mathscr{F}_{start}, \mathscr{F}_{end})} TPP(w) C(w, \textbf{x})
\end{equation}


\begin{equation}
\begin{array} {lllrrr} 
\displaystyle \operatorname*{arg\,min}_\pi & \displaystyle\sum_{i = 1}^n C_{\pi}(x_{ij})x_{ij} \\\\
& \displaystyle\sum_{n = 1}^N \sum_{k = 1}^K \sum_{x \in C} T(x)^{f_k^i} Y_x^{f_k^i} & \leq &  D_{\textbf{F}} \\
& \displaystyle\sum_{j = 1}^{n_i} x_{ij} & = & 1 & \forall i \in \lbrace 0, n \rbrace \\\\
& x_{ij} \in \lbrace 0, 1 \rbrace &&
\end{array}
\end{equation}


















\section{Super-peer Node}


An \textit{end-user} represents, instead, a third-party application that wants execute  one or more function choreographies.

\section{Peer Node}

Proposed 

Hybrid structures are notably deployed in collaborative distributed systems.
The main issue in many of these systems is to first get started, for which often
a traditional client-server scheme is deployed. Once a node has joined the
system, it can use a fully decentralized scheme for collaboration.


Relating to a specified function choreography $X$ belonging to resource owner $R$, a peer $P$ of our system can be in one of the following states:

\begin{description}
\item[Active State] When $P$ has been marked as responsible for manage all invocation requests of $X$ forwarded by end users.
\item[Forwarder State] Otherwise
\end{description}


function choreographies (FCs) or workflows of
functions. 

As known, in server-less computing platforms, computation is done in \textbf{function instances}. These instances are completely managed by the server-less computing platform provider (SSP) and act as tiny servers where a function is been executed.


\section{Resources}





\section{System's resources and actors}



Given a resource owner $R$, there are two type of resources which he can manage:

\begin{enumerate}
\item Function choreographies.
\item Server-less function implementations (also called \textit{concrete server-less functions})
\end{enumerate}



\subsection{Server-less function swarms}

Informally, a so-called \textit{server-less function swarm} represent a set of concrete server-less function with very specific properties.

Precisely, let $l \in \mathbb{N}$ such that $l \neq 0$, $R$ a resource owner, $P$ a server-less computing platform provider and  $X_R$ a set of concrete server-less functions. Moreover, let $\textbf{X}_R$ the set containing all concrete function implementations defined and deployed by $R$ on any provider. 

A set $X_R \subseteq \textbf{X}_R$ is called a \textit{server-less function swarm}, or simply \textit{swarm}, if:

\begin{enumerate}
\item $|X_{R}| = n \geq 1$, that is $X_{R}$ must contain at least one concrete function.
\item $X_{R}$ contain concrete functions that share the same platform provider $P$ where they will be executed.
\item $X_{R}$ contain concrete functions that share the same limit $l$ in term of max number of server-less function instance runnable at the same time by the corresponding platform provider $P$. That limit is also called \textit{server-less function swarm's concurrency limit}, or simply, \textit{concurrency limit}. 
\end{enumerate}

Is very important to make clear that only at most $l$ concrete functions belonging to $X_R$ can be executed simultaneously by $P$. The value of $l$ depends by specific policies adopted by $P$; some of them imposed that limit \textit{per-account}, others \textit{per-functions}. Our model supports both approaches because:

\begin{itemize}

\item If $P$ imposed limits \textit{per-function}, then $|X_{R}| = 1$, that is, $X_{R}$ will contain only one function defined and deployed by $R$ in $P$, where $l$ will be represent the provider's per-function limit.

\item If $P$ imposed a limit \textit{per-account}, then generally $|X_{R}| \geq 1$ and it include all concrete server-less function deployed on $P$ while $l$ will be represent the provider's global limit. 
\end{itemize}

\subsubsection{Server-less function sub-swarms}

A sub-swarm of ${X_{R}}$, which we will denote with $\Delta_{X_{R}}$, is the term with which we denote any element belong to the power set\footnote{The \textit{power set} $\mathcal{P}(S)$ of a set $S$ is the set of all subsets of $S$, including the empty set and $S$ itself.} of $X_{R}$, excluding the empty set. Formally, any $\Delta_{X_{R}} \in \mathcal{P}(X_{R}) \setminus \oslash$ is a sub-swarm.

Is very important to remember that in our model any sub-swarm $\Delta_{X_{R}}$ of ${X_{R}}$ has the \textit{same} concurrency limit of ${X_{R}}$.

\section{Function choreography scheduler}

\subsection{Schedulability condition}

Let $FC_R$ a function choreography belonging to a resource owner $R$ and $F_{abstract}$ its server-less abstract functions set. Moreover, be $\textbf{X}_R$ the set containing all functions deployed by $R$ in any provider.

In order to effectively start the execution of a function choreography, is required that for each abstract function $f_{abstract} \in F$ \textit{at least one} concrete function $f$, which implements it, exists.

Formally, a function choreography is said \textit{schedulable} when: 
\begin{equation}
\label{eqn:SchedulabilityConditionOne}
\begin{array}{lc}

& \forall f_{abstract} \in F_{abstract} \\

FC_R \text{ is schedulable } \Leftrightarrow & \\
 & \exists f \in \textbf{X}_R \mid  f \text{ implements } f_{abstract} \\
\end{array}
\end{equation}

Although it is correct, the condition expressed by equation \ref{eqn:SchedulabilityConditionOne} is not very precise, because $\textbf{X}_R$ can contain some functions that doesn't implement any $f_{abstract} \in F_{abstract}$.

Therefore, we define $\Omega_{FC_R}$ as the set containing only concrete functions that are needed to execute $FC_R$, which can both to belong to any provider and to have different concurrency limits.

Since multiple implementations of a same abstract function can exist at the same time, we can exploit the notion of swarm and sub-swarm to formally define the set $\Omega_{FC_R}$.

Let $n \in \mathbb{N}$, such that $n \geq 1$, and $X_{R_i}$ the $i$-th swarm and $\Delta_{X_{R_i}}$ its sub-swarm which contains only concrete functions implementing one or more $f_{abstract} \in F_{abstract}$, where $1 \leq i \leq n$.

We define $\Omega_{FC_R}$ as follows: 

\begin{equation}
\begin{array}{c}
\Omega_{FC_R} \mathDef \Delta_{X_{R_1}} \cup  \ldots \cup \Delta_{X_{R_n}} = \bigcup_{i = 1}^n \Delta_{X_{R_i}} \\

\text{ where } \\	

\Delta_{X_{R_i}} \cap \Delta_{X_{R_j}} = \oslash \quad \text{ for } i,j \in \left[ 0, n \right] \mid i \neq j \\

\forall f \in \Delta_{X_{R_s}} \quad f \text{ implements } f_{abstract} \text{ for } s \in \left[ 0, n \right] \\

\end{array}
\end{equation}

Since belong to different swarms, please note that any $\Delta_{X_{R_i}}$ and $\Delta_{X_{R_j}}$, for any $i \neq j$, can belong to the same provider but they cannot share the same concurrency limit.

Generally the schedulability condition for $FC_R$ can be written as follows:

\begin{equation}
FC_R \text{ is schedulable } \Leftrightarrow \exists \Omega_{FC_R}
\end{equation}


\subsection{The $\Delta_{X_{R}}$-Scheduler}

Let $R$ a resource owner, $P$ the server-less provider and $\Delta_{X_{R}}$ a sub-swarm of a $X_{R}$, where $k$ its concurrency limit.

Let $m \in \mathbb{N}$, a \textit{$\Delta_{X_{R}}$-Scheduler}, denoted as $S_{(\Delta_{X_{R}},m)}$ represents a queuing system, implementing any scheduling discipline, equipped with $m$ so-called \textit{virtual function instance}, where $m \leq k$.

Its aim is to decide when and which function, belonging to $\Delta_{X_{R}}$, must be performed on $P$. 

The parameter $m$ is also called \textit{scheduler capacity}.

\subsubsection{Virtual function instance}

A \textit{virtual function instance} represents a real function instances, clearly belonging to the server-less computing platform provider, which is \textit{virtually} owned by $S_{(\Delta_{X_{R}},m)}$.

Therefore, $m$ represents the max number of server-less function instances usable simultaneously by $S_{(\Delta_{X_{R}},m)}$.

\subsubsection{Proprieties and constrains}

According to our model, a $\Delta_{X_{R}}$-scheduler capable to manage any function belonging to $\Delta_{X_{R}}$, if exist, is \textit{not} unique, although it is unique inside a peer node. 

In order to achieve better performance in terms of network delay experienced by end users, fault tolerance and load balance, any peer nodes can hold a $\Delta_{X_{R}}$-scheduler in order to manage incoming request sent by several users spread in different geographic regions. 

However, despite there is no upper bound to the number of $\Delta_{X_{R}}$-schedulers existing at the same time in our system, there is a limitation regarding the scheduler capacity of each existing scheduler. 

Let's start summarizing all rules regarding $\Delta_{X_{R}}$-schedulers:

\begin{enumerate}

\item All peer node of our system can hold a $\Delta_{X_{R}}$-scheduler object.

\item Each node can hold only one instance of type $\Delta_{X_{R}}$-scheduler.

\item Let $n \in \mathbb{N}$ such that $n \geq 1$, suppose that our system contains $n$ peer nodes holding a $\Delta_{X_{R}}$-scheduler.

To be more precise, let's say that a sequence $S_{(1,(\Delta_{X_{R}},m_1))}, \ldots , S_{(i,(\Delta_{X_{R}},m_n))}$ exist at the same time in our system, where $S_{(i,(\Delta_{X_{R}},m_i))}$ represent the $\Delta_{X_{R}}$-scheduler owned by $i$-th node having scheduler capacity equal to $m_i$.

Following constraint must be hold:

\begin{equation}
\label{eqn:SchedulerConstrains}
\sum_{i=1}^{n} m_i \leq k
\end{equation}

where $k \in \mathbb{N}$, with $k > 0$, is the concurrency limit of the swarm $X_R$.

Remember that any sub-swarm $\Delta_{X_{R}}$ share the same concurrency limit of $X_R$. Therefore, equation \ref{eqn:SchedulerConstrains} states that, the sum of all scheduler capacities which manage the functions belonging to $\Delta_{X_{R}}$, must be less or equal to the max number of function instances executable at the same time on the server-less computing platform provider.

\end{enumerate}

\subsection{The $FC_R$-Scheduler}

To support hybrid-scheduling, that is the ability to execute multiple concrete function implementations belonging to different providers or subjected to different concurrency limit, in order to select the most suitable concrete function implementation according to a given QoS, unfortunately only one ``\textit{scheduler}" is not enough.

We call \textit{$FC_R$-Scheduler} a set of \textit{$\Delta_{X_{R}}$-Schedulers} where $\Delta_{X_{R}} \in \Omega_{FC_R}$. Is always required that $|FC_R| \geq 1$, that is, at least one scheduler must exist.

\section{$FC_R$-Active Peer Node}

According to our model, in order to effectively invoke all server-less concrete function belonging to a function choreography $FC_R$, is required that a peer node is ``active".

We said that a peer node $A$ is $FC_R$-\textit{active peer node}, or, simply, \textit{active}, when it holds a $\Delta_{X_{R}}$-scheduler for any sub-swarm in $\Omega_{FC_R}$. Formally:

\begin{equation}
A \text{ is } FC_R\text{-active peer node } \Leftrightarrow \forall \Delta_{X_{R}} \in \Omega_{FC_R} \quad \exists S_{(\Delta_{X_{R}},m)} \text{ hold by } A
\end{equation}

Multiple nodes can be active at the same time. Any node perform its scheduling decision independently.


\section{Architecture overview}

Our system design is based on a network of nodes, or \textit{peers}, every of which has the \textit{same functionality}; in fact, any of them is able to handle request submission, request scheduling and, potentially, request execution. 

This is the reason according to which we can mark our proposal as a \textit{P2P system}.

\subsection{Overlay network}

Our system's nodes are connected by an \textit{overlay network}.

\begin{definition}[Overlay network]
An overlay network is a virtual network built on top of a physical network according to which each nodes can communicate with other if and only if they are connected by virtual links belonging to the virtual network. 
\end{definition}

\begin{remark}
A node may not be able to communicate directly with an arbitrary other node although they can communicate through physical network.
\end{remark}

To be more precise, we have adopt a fully \textit{centralized unstructured overlay network} because the \textit{peer-resource index}, sometimes called \textit{directory}, is centralized. 

Please note that hybrid unstructured or fully decentralized solutions are technically possible, but guarantees about quality of service are very difficult.

\subsubsection{Locality-awareness property}

\textit{Locality-awareness} is one of the essential characteristics of our system. In fact, if each peer is able to select his neighbours exploiting a suitable locality aware algorithm, is possible to decrease user experienced delays.

We have decided to adopt a multi-level based locality-aware neighbour selection called \textit{intra-AS lowest delay clustering algorithm} (ASLDC). 

When ASLDC algorithm is used, each peer chooses nearby peers only from those within the same AS; then it ranks its neighbours in terms of transmission latency, preferring to o establish the connection with the node with the shortest latency to itself. 

TODO \cite{UnderstandingLocalityAwareness}

\subsection{Fault Tolerance}

Like in many other P2P implementation, there are two ways of detecting failures in our proposed solution:

\begin{enumerate}
\item If a node tries to communicate with a neighbour and fails.
\item Since all nodes send to all his neighbour nodes so-called ``heartbeat" messages, that is messages sent at fixed time intervals to indicate that the sender is alive, is possible to detect a failure by not receiving aforementioned periodic update messages after a long time. 

\end{enumerate}

\subsubsection{Availability}

Let $n \in \mathbb{N}$, $R$ a resource owner and $FC_R$ his function choreography, since multiple $FC_R$-Scheduler object can exist in $n$ different nodes, our proposed solution can guarantee an high degree of availability. 

In fact, when a node node holding an $FC_R$-Scheduler fails, all end users requests related to $FC_R$ can be routed to any other node holding a $FC_R$-Scheduler. 

\subsubsection{Replication}

Although multiple $FC_R$-Scheduler object can exist, it's not mean that $FC_R$-Scheduler object are replicated, because every of them manage different virtual function instances. 

Except the peer-resource index, any form of replication is performed by our system.







\begin{algorithm}
\caption{An algorithm with caption}\label{alg:cap}
\begin{algorithmic}

\State $y \gets 1$


\end{algorithmic}
\end{algorithm}




 and $\Delta_{X_{R}}$-Scheduler








The locality awareness of the overlay network is used in scheduling jobs, described in Section 3.2







physical network but with added properties such as fault tolerance and flexibility. 











The first problem is scheduling. Since there
is no central scheduler it is difficult



Hybrid unstructured overlay


\subsection{$FC_R$-Request}







 object with only one limitation:

Suppose that globally there are a set of schedulers $S_{1,({R_{X}},m_1)}, \ldots , S_{p,({R_{X}},m_p)}$, where $p \in \mathbb{N}$ with $p \geq 1$



To be more precise, when a function $x_j$ must to be execute, let $s$ the current number of busy virtual instances, one of the following events may occur:
\begin{enumerate}
\item if $s < m$, the scheduler invoke directly the function $x_j$ on the provider.
\item if $s = m$, the scheduler delay the execution of the function $x_j$ on the provider according to implemented scheduling discipline.
\end{enumerate}








Let $R$ a resource owner and $R_x$ its function choreography made up of $R_{x_1}, R_{x_2}, \ldots, R_{x_n}$ unique server-less functions; it is said that a peer node $P$ is \textbf{responsible} for $R_x$ when it contains a sequence of schedulers $S_{R_1}, S_{R_2}, \ldots, S_{R_k}$ with $k \leq n$, belonging to $R$, capable to invoke all server-less function belonging to $R_x$. 

It is said that a 

Depending on the definition of the function choreography provided by $R$ and the unique characteristics of back-end server-less providers which execute all serverless functions $R_{x_n}$ of 




It is said that a scheduler $S$ is capable to invoke a server-less function when 
, a scheduler $S$ can invoke multiple




When a peer $A$, placed ``\textit{at the edge}" of the network, receives a new request of invocation for $X$ by an end user, it performs following task in that order:

\begin{enumerate}
\item If it responsible It check for it is an already an \textit{active peer} to manage 
\end{enumerate}




 has found the tracker for a file F, the tracker returns a subset
of all the nodes currently involved in downloading F.




Replication and Fault Tolerance. There are two ways of detecting failures in CAN,
the first if a node tries to communicate with a neighbor and fails, it takes over that neighbor’s
zone. The second way of detecting a failure is by not receiving the periodic update message
after a long time. In the second case, the failure would probably be detected by all the
neighbors, and all of them would try to take over the zone of the failed node, to resolve this,
all nodes send to all other neighbors the size of their zone, and the node with the smallest
zone takes over.

\newpage


\begin{equation}
E[T] = \sum_{i = 0}^n E[S_i] + E[T_{Q_i}]
\end{equation}

\section{Queuing system of a $\Delta_{X_{R}}$-scheduler}

Let $k \in \mathbb{N}$ such that $k \geq 0$. Moreover, suppose that $R$ represents a resource owner, $P$ a server-less provider and $\Delta_{X_{R}}$ a sub-swarm of the swarm $X_{R}$ where $k$ its concurrency limit. Finally, let $S_{(\Delta_{X_{R}},m)}$ a $\Delta_{X_{R}}$-scheduler. 

Obliviously, since there are only $m$ available virtual function instances, if there are more than $m$ server-less functions waiting to be execute, a choice has to be made about which server-less function has to run next to ensuring QoS guarantees for latency critical applications.

Because we expect to execute server-less functions with different response-time requirements, which may have different scheduling needs, i have designed the $\Delta_{X_{R}}$-scheduler as a queuing system implementing a \textit{multilevel queue scheduling algorithm} which partitions the ready queue into several separate queues. 

According to our solution, each queue has its own scheduling algorithm and any server-less function is permanently assigned to one queue according to his class.

In addition, we have adopt \textit{round-robin} (\textit{RR}) scheduling algorithm to perform scheduling activity among the queues.

\subsection{Server-less function preemption}

A very important consideration regards server-less function preemption. 

In our context, due to FaaS paradigm, which hides the complexity of servers where our functions will be executed, the ability to preempt functions is not naturally available.

For that reason, only non-preemptive scheduling algorithms, according to which once a server-less function starts running it cannot be preempted, even if a higher priority server-less function comes along, can be adopted.

Since many scheduling algorithms require job preemption to run optimally, this situation can lead to a suboptimal resource management. 

\subsection{Queuing system design}

A $\Delta_{X_{R}}$-scheduler is made up of three queues:

\begin{itemize}

\item One queue implements a \textit{Non-Preemptive Least-Slack-Time-First} (\textit{LST}) scheduling algorithm. We will refer to that queue using $Q_{LST}$ notation.

That algorithm implements a \textit{dynamic priority scheduling approach} where priorities are assigned to server-less functions based on their \textit{slacks}.

At any time $t$, the \textit{slack} of a job with deadline at $d$ is equal to $d - t - s$, where $s$ is the time required to complete the job. 

Any job having strict latency requirements must to be assign to this queue.

\item Another queue implements, instead, a \textit{Non-Preemptive Shortest-Job-First} (\textit{SJF}) scheduling algorithm. That queue is denoted with $Q_{SJF}$.

That policy assigns priorities to jobs based on their size: the smaller the size, the higher the priority.

This queue is reserved for any function choreography that tolerates high latency. However, according to queueing theory, since SJF performs very bad when service time distribution is heavy-tailed; in fact

\begin{equation}
E[T_Q(x)]^{SJF} = \dfrac{\rho E[S^2]}{2E [S]} \cdot \dfrac{1}{(1 - \rho x )^2}
\end{equation}

This queue is not suitable for very, very large server-less function. The variance in the job size distribution must be low.

\item Finally, the third queue exploits the the simplest scheduling algorithm, that is the \textit{First-Come-First-Served} (\textit{FCFS}) policy, and it is denoted using $Q_{FCFS}$ notation.

This queue is used for any heavy-tailed server-less function which can tolerate high latency.
\end{itemize}


\subsection{The virtual function instances allocation problem}

Let $n \in \mathbb{N}$ such that $n \geq 1$ and suppose that, at a certain time, a sequence including $n$ unique $\Delta_{X_{R}}$-schedulers exist in our system.

Formally, that sequence is denoted with $S_{(1,(\Delta_{X_{R}},m_1))}, \ldots , S_{(i,(\Delta_{X_{R}},m_n))}$, where $S_{(i,(\Delta_{X_{R}},m_i))}$ represents the $\Delta_{X_{R}}$-scheduler owned by $i$-th node having scheduler capacity equal to $m_i$.

We have already established that the following constraint must be hold:

\begin{equation}
\label{eqn:SchedulerConstrains}
\sum_{i=1}^{n} m_i \leq k
\end{equation}

where $k \in \mathbb{N}$, with $k > 0$, is the concurrency limit of the swarm $X_R$.


\subsubsection{Reactive scaling policy}

Every node of the system monitors all queues belong to its $\Delta_{X_{R}}$-Scheduler and adjusts the number of reserved virtual function instances to preserve quality-of-service guarantees. This activity is called \textit{reactive scaling policy}.

Although a $S_{(\Delta_{X_{R}},m)}$ owns $m$ virtual function instances, they are not equally shared between aforementioned run queue. 

First of all, let be $v_{LST}$, $v_{SJF}$ and $v_{FCFS}$ the numbers of the virtual function instances bounded to the $Q_{LST}$, $Q_{SJF}$ and $Q_{FCFS}$ run-queue. 

At any time, following constrains must be hold:

\begin{equation}
v_{LST} \mathDef \dfrac{N \cdot E[S]}{\sum slack} 
\end{equation}

\begin{equation}
v_{SJF} \mathDef \dfrac{len(Q_{SJF})}{\alpha} 
\end{equation}

\begin{equation}
v_{FCFS} \mathDef \dfrac{len(Q_{FCFS})}{\alpha} 
\end{equation}

\begin{equation}
m = v_{LST} + v_{SJF} + v_{FCFS}
\end{equation}

\begin{enumerate}
\item If the current number of virtual function instances is not enough to avoid deadline misses by latency constrained server-less function inside $Q_{LST}$, will be add to $VCPU_{LST}$ a number of virtual processor equal to 

\begin{equation}
v_{LST} < \dfrac{N \cdot E[S]}{\sum slack} 
\end{equation}


\end{enumerate}



This form of aging prevents starvation. 


Finally, suppose that a node peer $A$ holds a $S_{(\Delta_{X_{R}},m_A)}$, which represents the 
$\Delta_{X_{R}}$-Scheduler object.

\begin{enumerate}
\item Once a request is received over HTTP
\item Ask to the coordinator for 
\end{enumerate}


\begin{equation}
m = \dfrac{\sum slack}{E[S] \cdot m_{current}} + \dfrac{N}{\alpha}
\end{equation}

\begin{equation}
\lambda_{FC_R} \geq \lambda_{threshold} 
\end{equation}


\begin{equation}
RTT_{(A,B)} > RTT_{(A,C)} + 
\end{equation}

\begin{equation}
f_{abstract} = next 
\end{equation}

\begin{equation}
f_{abstract} = min(E[f_{\Delta_{X_{R_i}}}] + E[T_{Q_{LST}}] F
\end{equation}






\subsubsection{Server-less function scheduling for QoS ``Minimum Response Time"}

Let $n \in \mathbb{N}$, such that $n \geq 1$, supposing to have an $FC_R$-\textit{active peer node} and let $\Omega_{FC_R}$ the set of all sub-swarms containing at least one function implementing any $f_{abstract} \in \textbf{F}_{abstract}$, where $|\Omega_{FC_R}| = n$.

Finally, suppose that $f_{abstract}$ is the next abstract function that have to be executed according to the control-flow logic described by $FC_R$.

\begin{enumerate}
\item For $1 \leq i \leq n$, let $\Delta_{(X_{R_i}, f_{abstract})} \subseteq \Delta_{X_{R_i}}$ containing all concrete functions implementing $f_{abstract}$.

\item For $1 \leq i \leq n$, select $\textbf{f}_i \in \Delta_{(X_{R_i}, f_{abstract})}$ functions having the minimum response time $E[T_{\textbf{f}_i}]$.

\item Found $i \in [0,m]$ such that: 

\begin{equation}
E[T_{\textbf{f}_i}] + E[T_{Q_i}]^{\textbf{LST}} < E[T_{\textbf{f}_j}] +E[T_{Q_j}]^{\textbf{LST}} \quad \text{ for } j \in \mathbb{N}, 1 \leq j \leq n, i \neq j
\end{equation}

\item Select the i-th \textit{$\Delta_{X_{R}}$-Schedulers} belonging to \textit{$FC_R$-Scheduler} of the node and add  $\textbf{f}_i$ into the $Q_{LST}$ run-queue.

\end{enumerate}




















\end{document}